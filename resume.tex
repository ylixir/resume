% Exemple de CV utilisant la classe moderncv
% Style classic en bleu
% Article complet : http://blog.madrzejewski.com/creer-cv-elegant-latex-moderncv/

\documentclass[letterpaper]{moderncv}
\moderncvtheme[blue]{classic}                
\usepackage[utf8]{inputenc}
\usepackage[top=1.1cm, bottom=1.1cm, left=2cm, right=2cm]{geometry}
% Largeur de la colonne pour les dates
\setlength{\hintscolumnwidth}{2.5cm}

\firstname{Jonathan}
\familyname{Allen}
\title{Mathematics and Software Development}              
\address{P.O. Box 52}{Fargo, ND 58107}    
\email{ylixir@gmail.com}                      
%\extrainfo{\emailsymbol\emaillink{ylixir@gmail.com}}
%\phone{802-552-0922} 
\mobile{802-552-0922} 
\homepage{https://github.com/ylixir}
\social[linkedin][https://www.linkedin.com/in/ylixir/]{https://linkedin.com/in/ylixir}
\begin{document}
\maketitle

\section{Objective}
Obtain a position which leverages my skills in software development, software architecture and mathematics, where I can learn and grow both professionally and intellectually.
\section{Education}
\cventry{2015}{North Dakota State University}{Fargo, ND}{}{}{
Bachelor of Arts in Mathematics
\begin{itemize}
\item
Elective credits in partial differential equations, combinatorics, graph theory and real analysis.
\item
Capstone project explored numerical semigroups, Markov bases, and extensions of the natural numbers.
\end{itemize}
}
\cventry{2001-2005}{Minot State University}{Minot, ND}{}{}{
Computer Science, Mathematics, Physics coursework
\begin{itemize}
\item
Met programming proficiency requirements.
\item
Completed all data structures and algorithms coursework.
\item
Completed a concentration in physics.
\end{itemize}
}

\section{Skills}
\cvitem{Expertise}{C/C++ (and family), sh, Lua, git, *nix}
\cvitem{Current Focus}{Elm, Haskell, Rust, Atmel AVR}

\section{Related Experience}
\cventry{2016 -- 2017}{Senior Software Developer}{RealTruck Inc.}{Fargo, ND}{}{Professional Software Development
\begin{itemize}
\item
Systems design and architecture.
\item
Code review of both peer and junior developers.
\item
Author technical guidelines.
\item
The ability to balance technical considerations with quarterly and political constraints.
\item
It so quickly became apparent how much of an asset experience in mathematics and open source software is that I was elevated to a senior level, despite being hired at a junior level.
\item
Ability to come up to speed with a nearly million line codebase in record time.
\item
The rigor and logic obtained from my mathematics experience allows me to find and fix bugs in days that other developers can not find in months.
\end{itemize}}
\cventry{1985 -- present}{Software Developer}{}{}{}{A lifetime of non-professional software development.
\begin{itemize}
\item
Breadth of experience crossing many paradigms and technologies.
\item
Ability to quickly learn and become adept at any "new" technology.
\item
Ability to evaluate best fit technologies, regardless of political or quarterly constraints.
\item
Evaluating long term value of technical decisions.
\end{itemize}}
\cventry{2014 -- 2015}{Teaching Assistant}{North Dakota State University}{Fargo, ND}{}{Precalculus level algebra
\begin{itemize}
\item
Prepare and present classroom material.
\item
Provide one-on-one assistance for my students.
\item
Grade homework, quizzes, exams, etc.
\end{itemize}}
\cventry{2003 -- 2016}{Manager}{}{Minot/Fargo, ND}{}{Food Service
\begin{itemize}
\item Papa John's and McDonald's
\item Food and inventory preparation and management.
\item Employee and labor management.
\item Cash management.
\item Employee training.
\item Customer service.
\end{itemize}}
\cventry{2002 -- 2003}{Programmer}{North Dakota Center for Persons with Disabilities}{Minot, ND}{}{Miscellaneous programming tasks
\begin{itemize}
\item
Use C++ and Win32 to maintain and create accessibility software.
\item
Use ASP to create and maintain web applications.
\item
Use SQL to interface with database back ends.
\item
Design and maintain databases using MSSQL and Oracle.
\end{itemize}
}
\section{Technical Achievements}
\cventry{2017}{Sales tax system}{}{https://www.realtruck.com}{}{
	Completely reimplemented the sales tax handling for major e-commerce platform. When I came on board taxes were hardcoded in stored procedures in our database. Taxes are now handled dynamically and intelligently. I had to write a custom SDK for integration with Avalara from scratch because of the technical limitations of our platform.
	\begin{itemize}
	\item Languages: PHP
	\item Technologies: JSON, HTTPS
	\end{itemize}
}
\cventry{2017}{Implement gift card system}{}{https://www.realtruck.com}{}{
	Political and technical decisions left our platform with no ability to handle gift cards. I implemented gift card integration with lightrail.com which allows use of previously purchased cards, and the ability to create and use new cards.
	\begin{itemize}
	\item Languages: PHP
	\item Technologies: JSON, HTTPS, ADTs
	\end{itemize}
}
\cventry{2016}{Complete redesign of ad feed system}{}{https://www.realtruck.com}{}{
	Refactored a system that was literally unmaintainable. This system was so "legacy" that noone else was able to work with it. It was described to me as "the worst part of our codebase" and was so brittle that any modification would likely cause the loss of tens of thousands of dollars in revenue. My ability to maintain, improve and test this system while balancing time and priorities of other tasks, was described as a "master class in incremental improvement". Currently the system can be easily modified without risks of side effects, and can be plugged into arbitrary web api technologies.
	\begin{itemize}
	\item Languages: PHP
	\end{itemize}
}
\cventry{2015}{diceware}{}{https://www.github.com/ylixir/diceware}{}{
	Utility for generating passphrases. These are very secure passwords, which are easy to remember.
	\begin{itemize}
	\item Languages: Lua
	\item Technologies: diceware, /dev/urandom
	\end{itemize}
}
\cventry{2016}{yotp}{}{https://www.github.com/ylixir/yotp}{}{
	Command line utility for generating one time passwords. These can be used, for example, to log into ones Facebook account after enabling two factor authentication. This code could also be used on the other end, by a web service to validate users using one time passwords.
	\begin{itemize}
	\item Languages: C\#
	\item Technologies: .net core, Mono, .NET, HOTP, TOTP, SHA1
	\end{itemize}
}
\cventry{2015}{frobmask}{}{https://www.github.com/ylixir/frobmask}{}{
	Automates computation of Frobenius numbers. Useful to mathematicians studying numerical semigroups.
	\begin{itemize}
	\item Languages: Lua 5.3
	\item Technologies: Abstract Algebra
	\end{itemize}
}
\cventry{2015}{Lerna}{}{https://www.github.com/ylixir/lerna}{}{
	Web browser with lua scripting support.
	\begin{itemize}
	\item Languages: Vala, Lua
	\item Technologies: GTK3, WebKit, liblua
	\end{itemize}
}
\cventry{2013}{ArchNexus}{}{https://www.github.com/archnexus}{}{
	Created Linux distribution for tablet computers.
	\begin{itemize}
	\item Languages: sh, C
	\item Technologies: Linux, gcc, pacman
	\end{itemize}
}
\cventry{2010 -- 2011}{Yaed}{}{https://www.github.com/ylixir/yaed}{}{
	Cross platform text editor. This was an interesting excercise in documentation first, code second, which worked out really well. This is the kind of thing I would love to see in a professional environment, but seems rare.
	\begin{itemize}
	\item Languages: C
	\item Technologies: GTK-2, GTK-3, GtkSourceView
	\end{itemize}
}
\cventry{2008 -- 2011}{yCurses}{}{https://www.github.com/ylixir/ycurses}{}{
	Created ncurses bindings for the D programming language. To be honest the thing I learned from this project is to not be stingy with my work, and to welcome collaborators. Unfortanately this project wound up forked because I did not welcome assistance from people with more bandwidth than me.
	\begin{itemize}
	\item Languages: C, D
	\item Technologies: nCurses
	\end{itemize}
}

\end{document}

