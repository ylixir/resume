% Exemple de CV utilisant la classe moderncv
% Style classic en bleu
% Article complet : http://blog.madrzejewski.com/creer-cv-elegant-latex-moderncv/

\documentclass[letterpaper]{moderncv}
\moderncvtheme[blue]{classic}                
\usepackage[utf8]{inputenc}
\usepackage[top=1.1cm, bottom=1.1cm, left=2cm, right=2cm]{geometry}
% Largeur de la colonne pour les dates
\setlength{\hintscolumnwidth}{2.5cm}

\firstname{Jonathan}
\familyname{Allen}
\title{Mathematics and Software Development}              
\address{P.O. Box 52}{Fargo, ND 58107}    
\email{ylixir@gmail.com}                      
%\extrainfo{\emailsymbol\emaillink{ylixir@gmail.com}}
%\phone{802-552-0922} 
\mobile{802-552-0922} 
\homepage{https://github.com/ylixir}
\social[linkedin][https://www.linkedin.com/in/ylixir/]{https://linkedin.com/in/ylixir}
\begin{document}
\maketitle

\section{Objective}
Obtain a position in which I am able to mentor and be mentored in order to expand skills in software development, software architecture and mathematics.
\section{Related Experience}
\cventry{2016 -- present}{Senior Software Developer}{RealTruck Inc.}{Fargo, ND}{}{Professional Software Development
\begin{itemize}
\item
Enterprise systems design and architecture.
\item
Code review of both peer and junior developers.
\item
Author technical guidelines.
\item
Balancing technical considerations with quarterly and political constraints.
\item
Applying experience gained from open source development to proprietary systems.
\item
Quickly coming to grips with very large codebases.
\item
Apply mathematical experience to finding and fixing seemingly intractable software bugs.
\item
Engineer for failure and recovery, providing robust and highly resilient systems.
\end{itemize}}
\cventry{1985 -- present}{Software Developer}{}{}{}{A lifetime of non-professional software development.
\begin{itemize}
\item
Breadth of experience crossing many paradigms and technologies.
\item
Ability to quickly learn and become adept at any "new" technology.
\item
Ability to evaluate best fit technologies, regardless of political or quarterly constraints.
\item
Evaluating long term value of technical decisions.
\end{itemize}}
\cventry{2014 -- 2015}{Teaching Assistant}{North Dakota State University}{Fargo, ND}{}{Precalculus level algebra
\begin{itemize}
\item
Prepare and present classroom material.
\item
Provide one-on-one assistance for my students.
\item
Grade homework, quizzes, exams, etc.
\end{itemize}}
\cventry{2002 -- 2003}{Programmer}{North Dakota Center for Persons with Disabilities}{Minot, ND}{}{Miscellaneous programming tasks
\begin{itemize}
\item
Diverse technology stacks consisting of C++, Win32 API, MFC, ASP, MSSQL, Oracle.
\item
Create and maintain desktop and web applications with a focus on accessibility software.
\end{itemize}
}

\section{Skills}
\cvitem{Expertise}{C/C++ (and family), JavaScript, git, *nix}
\cvitem{Current Focus}{Elm, Haskell, Rust, Atmel AVR}

\section{Education}
\cventry{2015}{North Dakota State University}{Fargo, ND}{}{}{
Bachelor of Arts in Mathematics
\begin{itemize}
\item
Elective credits in partial differential equations, combinatorics, graph theory and real analysis.
\item
Capstone explored numerical semigroups, Markov bases, and extensions of the natural numbers.
\end{itemize}
}
\cventry{2001-2005}{Minot State University}{Minot, ND}{}{}{
Computer Science, Mathematics, Physics coursework
\begin{itemize}
\item
Exempted from basic programming (C++) coursework.
\item
Completed all data structures and algorithms coursework.
\item
Completed a concentration in physics.
\end{itemize}
}
\section{Selected Technical Achievements}
\cventry{2017}{Sales tax system}{}{https://www.realtruck.com}{}{
	Hardcoded taxe rates weren't scaling. Technical limitations of our platform forced me to write from scratch a custom SDK that integrated our platform with Avalara. Taxes are now calculated intelligently and dynamically.
	\begin{itemize}
	\item Languages: PHP
	\item Technologies: JSON, HTTPS, REST
	\end{itemize}
}
\cventry{2017}{Implement gift card system}{}{https://www.realtruck.com}{}{
	Political and technical decisions left our platform with no ability to handle gift cards. I implemented gift card integration with Lightrail.com, restoring gift card capabilities to our platform.
	\begin{itemize}
	\item Languages: PHP
	\item Technologies: JSON, HTTPS, ADTs
	\end{itemize}
}
\cventry{2016}{Complete redesign of ad feed system}{}{https://www.realtruck.com}{}{
	Refactored a system described as "the worst part of our codebase". Mistakes made modifying this system routinely caused the loss of tens of thousands of dollars. Maintaining, improving and testing this system while balancing time and priorities of other tasks, was described by management as a "master class in incremental improvement". The system can now be easily modified without risks of side effects or failure, and can be plugged into arbitrary web technologies for consumption by advertising partners.
	\begin{itemize}
	\item Languages: PHP
	\end{itemize}
}
\cventry{2016}{yotp}{}{https://www.github.com/ylixir/yotp}{}{
	Command line utility for generating one time passwords. Commonly called two factor authentication, this code could be used by a client or server to implement this technology.
	\begin{itemize}
	\item Languages: C\#
	\item Technologies: .net core, Mono, .NET, HOTP, TOTP, SHA1
	\end{itemize}
}
\cventry{2015}{diceware}{}{https://www.github.com/ylixir/diceware}{}{
	Utility for generating passphrases. These are very secure passwords, which are easy to remember.
	\begin{itemize}
	\item Languages: Lua
	\item Technologies: diceware, /dev/urandom
	\end{itemize}
}
\cventry{2015}{frobmask}{}{https://www.github.com/ylixir/frobmask}{}{
	Automates computation of Frobenius numbers. Useful to mathematicians studying numerical semigroups.
	\begin{itemize}
	\item Languages: Lua 5.3
	\item Technologies: Abstract Algebra
	\end{itemize}
}
\cventry{2015}{Lerna}{}{https://www.github.com/ylixir/lerna}{}{
	Web browser with lua scripting support.
	\begin{itemize}
	\item Languages: Vala, Lua
	\item Technologies: GTK3, WebKit, liblua
	\end{itemize}
}
\cventry{2013}{ArchNexus}{}{https://www.github.com/archnexus}{}{
	Created Linux distribution for tablet computers.
	\begin{itemize}
	\item Languages: sh, C
	\item Technologies: Linux, gcc, pacman
	\end{itemize}
}
\cventry{2010 -- 2011}{Yaed}{}{https://www.github.com/ylixir/yaed}{}{
	Cross platform text editor. This was an interesting excercise in documentation first, code second, which worked out really well. This is the kind of thing I would love to see in a professional environment, but seems rare.
	\begin{itemize}
	\item Languages: C
	\item Technologies: GTK-2, GTK-3, GtkSourceView
	\end{itemize}
}
\cventry{2008 -- 2011}{yCurses}{}{https://www.github.com/ylixir/ycurses}{}{
	Created ncurses bindings for the D programming language. To be honest the thing I learned from this project is to not be stingy with my work, and to welcome collaborators. Unfortanately this project wound up forked because I did not welcome assistance from people with more bandwidth than me.
	\begin{itemize}
	\item Languages: C, D
	\item Technologies: nCurses
	\end{itemize}
}

\end{document}

